\chapter{Reflection and Learning}

At the outset, the undertaking of this project was quite daunting, partially due to its inherent complexity, tackling a topic far beyond the technical scope of any of the taught modules covered in the MSc. I was therefore acutely aware that I was to be challenging myself considerably, both technically and academically, requiring me to learn at pace to deliver on my research aims.

The process of delivering this project has given me valuable insight into the breadth of skills required to deliver a program of research. It very quickly became apparent that time management and organisation skills were essential, along with the ability to be agile with ones aims and objectives. 

The use of a Gantt chart was helpful in enabling a broad overview of progress. However, I quickly discovered that when embarking on a novel research area that is inherently complexed and challenging to both one’s technical skills and domain knowledge, the intricacies of the tasks can require altering priorities at short notice. For example, running the required volume of iterations of the network could take anywhere between 2-3 days depending on how many permutations and training variables were altered. If a bug appeared in the the code, particularly one that I was not aware of due to the use of a remote machine, my workflow would have to pivot to allowing time for debugging. This made planning unwieldy at times, as each collation of results required data analysis from which I used to inform on the next permutations of the networks. 

In an ideal world the data analysis tools and skills to monitor the CNNs output would be a prerequisite before commencing with the research. However, the bespoke data analysis tools required for the task were not available from the outset, as I was not yet equipped with the skills and agility required to build them. As such, it initially took substantial time to complete analysis, using a combination of excel and Jupyter Notebooks, before fully transitioning to the latter. This limitation in my skill set served as a vital learning curve, as the desire to complete analysis at pace, drove me to learn a host of new skills in data manipulation. The acquisition of these new analysis skills, particularly in the visual format, served to reinforce my knowledge of data visualisation acquired during the taught phase of the MSc. The subsequent understanding from these visual aids made me vastly more agile, enabling me to hypothesise on the implications of my network alterations, and to re-aligning my objectives to fit within the limited time frame of the project. 

Initially the ambiguity of the time frames required for the network to run and the data to be analysed often meant that I found myself neglecting the Gantt chart and using a combination of handwritten notes and task lists, word documents comprised of hypotheses, and technical summaries, windows notepads for reminders of which permutation of the model was running on the remote machine, alongside a host of annotations in the code. As the whole process of data collection, analysis, literature acquisition and review became more fluid, I found myself returning to  the Gantt chart as means of keeping on top of progress.
In hindsight, I vastly overestimated my ability to complete the scale of research that I desired in such a short time frame. The discovery midway through the project that the manipulation of training options was not necessarily the best means to achieve increased accuracy, at least in this instance, redefined the scope of the project. Subsequently I made the executive decision to pivot my attention to analysing the effect of removing a weakly correlated feature concept and of altering concepts ratings as binary or scalar. Prior knowledge of neural network optimisation methods before undertaking this project would have saved a great deal of time and effort, however I learned a great deal in the process. I am now considerably more informed and skilled as making alterations to the inner working of CNNs and AI systems in general, of which I hope to apply in my work beyond this project. 

At the start of the project I was only briefly acquainted with Matlab, having used it for assisting co-workers in research activities. The subsequent development of the project in Matlab gave me a far greater understanding of the language and environment for the undertaking of research and development work. Despite this, it did hinder my ability to experiment with new methodologies and tools discovered in my literature search, of which the majority were typically developed in Python (and freely available on GitHub). Given the time and the opportunity I would preferred to have re-written the project in Python, and to have incorporated a means of using Shapley values \cite{yehCompletenessawareConceptBasedExplanations2019} to assess the model’s dependence on a concept/feature.

Due to my involvement in the conference paper \cite{grangeXAISelfexplanatoryAI2022} that inspired this dissertation, I was put in good stead to further embark on a wider literature review, to inform the necessity and relevancy of my work within the academic landscape. My thirst for knowledge and enthusiasm for discovering new methods and ideas in XAI put my time management skills to the test when undertaking the acquisition of literature for review. Upon reflection, I believe that my compulsion to understand the broader narrative as well as the granular, was essential, as it became apparent from reading the vast platitude of papers published on AI and XAI, that few in the field fully consider the inherent interdisciplinary nature of their work and the broader landscape in which it is sited e.g. psychology and philosophy.

The discovery of papers on covering self-explanatory or interpretable XAI models and methods that were conceptually relevant was a complexed task. The terminology used by authors is frequently used interchangeably, with multiple interpretations of their meanings thus posing a great challenge when making comparative evaluations of models. As such I decided to include a subsection within this dissertation addressing the issue, which not only aided my understanding, but hopefully the readers too.

The lack of commonly used methodologies by which to assess XAI approaches empirically was also preventative in assessing XAI models on any common ground. The field of Human Factors, HCI and Robotics have many tools at their disposal with which to use for empirical research, however those in XAI often appended their own methods for assessing the credibility of their work.  I sincerely hope that this will soon change with the very recent publication of papers addressing the issue (\cite{hoffmanMeasuresExplainableAI2023}, \cite{uenoTrustHumanAIInteraction2022}, \cite{perrigTrustIssuesTrust2023}, \cite{kleinMinimumNecessaryRigor2023}).

As my literature review was ongoing throughout the project, with my task of developing the hybrid sequential network already set out at the beginning of the project, it was near impossible to back track and incorporate any of the new tools that I subsequently discovered in the literature. The discovery of Koh et al.’s \cite{kohConceptBottleneckModels2020} paper was somewhat disheartening, with the proposal of a sequential CBM vastly akin to that proposed by Grange et al. \cite{grangeXAISelfexplanatoryAI2022}. However the lack of grounding in the psychological domain i.e. the research conducted by Nosofsky et al. set’s the papers apart significantly. The discovery Koh et al.’s paper did however inform me of the term “Concept Bottleneck Network”, of which greatly assisted my literature search due its adoption by subsequent researchers.

As such, the project has up-skilled me immensely, with an insight and knowledge to build my own tools and tackle debates over the preference of one methodology over another.

Throughout the project I lacked the benefit of garnering any insight for external experts (except at its inception) or empirical evidence with which to evaluate the validity of my models output e.g. conferring with geologists over the concepts used or measuring the skills of geologists to correctly classify the rocks by image alone (typically done by a variety of physical methods rather than visual alone). 

lessons about the topics addressed in the project where not already covered by the substance of your dissertation (underpinning theory or philosophy; value of approaches; understanding gained; problems not solved; effectiveness; etc.). 